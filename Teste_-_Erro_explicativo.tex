\documentclass[]{article}
\usepackage{lmodern}
\usepackage{amssymb,amsmath}
\usepackage{ifxetex,ifluatex}
\usepackage{fixltx2e} % provides \textsubscript
\ifnum 0\ifxetex 1\fi\ifluatex 1\fi=0 % if pdftex
  \usepackage[T1]{fontenc}
  \usepackage[utf8]{inputenc}
\else % if luatex or xelatex
  \ifxetex
    \usepackage{mathspec}
  \else
    \usepackage{fontspec}
  \fi
  \defaultfontfeatures{Ligatures=TeX,Scale=MatchLowercase}
\fi
% use upquote if available, for straight quotes in verbatim environments
\IfFileExists{upquote.sty}{\usepackage{upquote}}{}
% use microtype if available
\IfFileExists{microtype.sty}{%
\usepackage{microtype}
\UseMicrotypeSet[protrusion]{basicmath} % disable protrusion for tt fonts
}{}
\usepackage[margin=1in]{geometry}
\usepackage{hyperref}
\hypersetup{unicode=true,
            pdftitle={Teste de erro explicativo},
            pdfauthor={Felipe N. S. Bezerra},
            pdfborder={0 0 0},
            breaklinks=true}
\urlstyle{same}  % don't use monospace font for urls
\usepackage{color}
\usepackage{fancyvrb}
\newcommand{\VerbBar}{|}
\newcommand{\VERB}{\Verb[commandchars=\\\{\}]}
\DefineVerbatimEnvironment{Highlighting}{Verbatim}{commandchars=\\\{\}}
% Add ',fontsize=\small' for more characters per line
\usepackage{framed}
\definecolor{shadecolor}{RGB}{248,248,248}
\newenvironment{Shaded}{\begin{snugshade}}{\end{snugshade}}
\newcommand{\KeywordTok}[1]{\textcolor[rgb]{0.13,0.29,0.53}{\textbf{#1}}}
\newcommand{\DataTypeTok}[1]{\textcolor[rgb]{0.13,0.29,0.53}{#1}}
\newcommand{\DecValTok}[1]{\textcolor[rgb]{0.00,0.00,0.81}{#1}}
\newcommand{\BaseNTok}[1]{\textcolor[rgb]{0.00,0.00,0.81}{#1}}
\newcommand{\FloatTok}[1]{\textcolor[rgb]{0.00,0.00,0.81}{#1}}
\newcommand{\ConstantTok}[1]{\textcolor[rgb]{0.00,0.00,0.00}{#1}}
\newcommand{\CharTok}[1]{\textcolor[rgb]{0.31,0.60,0.02}{#1}}
\newcommand{\SpecialCharTok}[1]{\textcolor[rgb]{0.00,0.00,0.00}{#1}}
\newcommand{\StringTok}[1]{\textcolor[rgb]{0.31,0.60,0.02}{#1}}
\newcommand{\VerbatimStringTok}[1]{\textcolor[rgb]{0.31,0.60,0.02}{#1}}
\newcommand{\SpecialStringTok}[1]{\textcolor[rgb]{0.31,0.60,0.02}{#1}}
\newcommand{\ImportTok}[1]{#1}
\newcommand{\CommentTok}[1]{\textcolor[rgb]{0.56,0.35,0.01}{\textit{#1}}}
\newcommand{\DocumentationTok}[1]{\textcolor[rgb]{0.56,0.35,0.01}{\textbf{\textit{#1}}}}
\newcommand{\AnnotationTok}[1]{\textcolor[rgb]{0.56,0.35,0.01}{\textbf{\textit{#1}}}}
\newcommand{\CommentVarTok}[1]{\textcolor[rgb]{0.56,0.35,0.01}{\textbf{\textit{#1}}}}
\newcommand{\OtherTok}[1]{\textcolor[rgb]{0.56,0.35,0.01}{#1}}
\newcommand{\FunctionTok}[1]{\textcolor[rgb]{0.00,0.00,0.00}{#1}}
\newcommand{\VariableTok}[1]{\textcolor[rgb]{0.00,0.00,0.00}{#1}}
\newcommand{\ControlFlowTok}[1]{\textcolor[rgb]{0.13,0.29,0.53}{\textbf{#1}}}
\newcommand{\OperatorTok}[1]{\textcolor[rgb]{0.81,0.36,0.00}{\textbf{#1}}}
\newcommand{\BuiltInTok}[1]{#1}
\newcommand{\ExtensionTok}[1]{#1}
\newcommand{\PreprocessorTok}[1]{\textcolor[rgb]{0.56,0.35,0.01}{\textit{#1}}}
\newcommand{\AttributeTok}[1]{\textcolor[rgb]{0.77,0.63,0.00}{#1}}
\newcommand{\RegionMarkerTok}[1]{#1}
\newcommand{\InformationTok}[1]{\textcolor[rgb]{0.56,0.35,0.01}{\textbf{\textit{#1}}}}
\newcommand{\WarningTok}[1]{\textcolor[rgb]{0.56,0.35,0.01}{\textbf{\textit{#1}}}}
\newcommand{\AlertTok}[1]{\textcolor[rgb]{0.94,0.16,0.16}{#1}}
\newcommand{\ErrorTok}[1]{\textcolor[rgb]{0.64,0.00,0.00}{\textbf{#1}}}
\newcommand{\NormalTok}[1]{#1}
\usepackage{graphicx,grffile}
\makeatletter
\def\maxwidth{\ifdim\Gin@nat@width>\linewidth\linewidth\else\Gin@nat@width\fi}
\def\maxheight{\ifdim\Gin@nat@height>\textheight\textheight\else\Gin@nat@height\fi}
\makeatother
% Scale images if necessary, so that they will not overflow the page
% margins by default, and it is still possible to overwrite the defaults
% using explicit options in \includegraphics[width, height, ...]{}
\setkeys{Gin}{width=\maxwidth,height=\maxheight,keepaspectratio}
\IfFileExists{parskip.sty}{%
\usepackage{parskip}
}{% else
\setlength{\parindent}{0pt}
\setlength{\parskip}{6pt plus 2pt minus 1pt}
}
\setlength{\emergencystretch}{3em}  % prevent overfull lines
\providecommand{\tightlist}{%
  \setlength{\itemsep}{0pt}\setlength{\parskip}{0pt}}
\setcounter{secnumdepth}{0}
% Redefines (sub)paragraphs to behave more like sections
\ifx\paragraph\undefined\else
\let\oldparagraph\paragraph
\renewcommand{\paragraph}[1]{\oldparagraph{#1}\mbox{}}
\fi
\ifx\subparagraph\undefined\else
\let\oldsubparagraph\subparagraph
\renewcommand{\subparagraph}[1]{\oldsubparagraph{#1}\mbox{}}
\fi

%%% Use protect on footnotes to avoid problems with footnotes in titles
\let\rmarkdownfootnote\footnote%
\def\footnote{\protect\rmarkdownfootnote}

%%% Change title format to be more compact
\usepackage{titling}

% Create subtitle command for use in maketitle
\newcommand{\subtitle}[1]{
  \posttitle{
    \begin{center}\large#1\end{center}
    }
}

\setlength{\droptitle}{-2em}

  \title{Teste de erro explicativo}
    \pretitle{\vspace{\droptitle}\centering\huge}
  \posttitle{\par}
    \author{Felipe N. S. Bezerra}
    \preauthor{\centering\large\emph}
  \postauthor{\par}
      \predate{\centering\large\emph}
  \postdate{\par}
    \date{20 de outubro de 2018}


\begin{document}
\maketitle

\begin{Shaded}
\begin{Highlighting}[]
\KeywordTok{set.seed}\NormalTok{(}\DecValTok{8}\NormalTok{)}
\end{Highlighting}
\end{Shaded}

Gerando valores aleatórios para X1

\begin{Shaded}
\begin{Highlighting}[]
\NormalTok{X1 <-}\StringTok{ }\KeywordTok{rnorm}\NormalTok{(}\DecValTok{50}\NormalTok{,}\DataTypeTok{mean=}\DecValTok{3}\NormalTok{,}\DataTypeTok{sd=}\DecValTok{10}\NormalTok{)}
\NormalTok{X1}
\end{Highlighting}
\end{Shaded}

\begin{verbatim}
##  [1]   2.1541393  11.4040013  -1.6348277  -2.5083500  10.3604043
##  [6]   1.9211860   1.2971085  -7.8833171 -27.1105168  -2.9317433
## [11]  -4.5979380   5.9204986   7.2139859  -9.9448908   3.6928509
## [16]  -5.1303848  18.1085307   0.2839112  18.5825393   0.6265013
## [21]  15.8312288   2.9051036  -1.0008278   3.2198562  20.4276738
## [26]  -8.0717416  -7.6048948  22.5116132   9.0270674 -17.2060948
## [31]  18.0667554  12.6371109 -12.5389141  -4.7364238  15.6077796
## [36]   7.2845578  11.4458354  -3.9915055   2.3513772   7.7098365
## [41]   6.6350062  -3.1455591   5.2894929   0.3794317   9.1255599
## [46] -16.5761249  -3.6350747   5.8378752   5.6871796  -3.4010474
\end{verbatim}

Gerando X2 com altamente correlacionado com X1

\begin{Shaded}
\begin{Highlighting}[]
\NormalTok{X2 <-}\StringTok{ }\DecValTok{15} \OperatorTok{-}\FloatTok{0.9}\OperatorTok{*}\NormalTok{X1 }\OperatorTok{+}\KeywordTok{rnorm}\NormalTok{(}\DecValTok{50}\NormalTok{,}\DataTypeTok{mean=}\DecValTok{0}\NormalTok{,}\DataTypeTok{sd=}\DecValTok{2}\NormalTok{)}
\NormalTok{X2}
\end{Highlighting}
\end{Shaded}

\begin{verbatim}
##  [1] 15.2211772  4.0061062 17.6500757 17.0653231  7.6186516 14.4278705
##  [7] 12.2782553 20.8933012 36.4008263 19.1468958 20.2527328  9.8613055
## [13]  4.9445298 21.7328169 11.1415386 17.4640401 -2.2799811 18.0228759
## [19] -0.9307313 11.2400020 -2.1834863 14.4656825 17.4664070 11.1479365
## [25] -3.1968572 23.4625261 16.0498094 -4.0930585 11.6278061 29.1929612
## [31] -0.7739419  2.0476138 24.0976238 23.3519961 -2.4096689 10.8744643
## [37]  4.5073011 16.6895660 14.0862915  2.0320927 10.5674636 20.9688872
## [43]  9.2340228 13.4227351  4.0841321 31.4566882 18.9305055  9.6576210
## [49]  8.6070638 17.9223772
\end{verbatim}

Regressão de X1 explicado por X2

\begin{Shaded}
\begin{Highlighting}[]
\KeywordTok{cor}\NormalTok{(X1, X2)}
\end{Highlighting}
\end{Shaded}

\begin{verbatim}
## [1] -0.969517
\end{verbatim}

\begin{Shaded}
\begin{Highlighting}[]
\NormalTok{mod.X1.X2 <-}\StringTok{ }\KeywordTok{lm}\NormalTok{(X1 }\OperatorTok{~}\StringTok{ }\NormalTok{X2)}
\KeywordTok{summary}\NormalTok{(mod.X1.X2)}
\end{Highlighting}
\end{Shaded}

\begin{verbatim}
## 
## Call:
## lm(formula = X1 ~ X2)
## 
## Residuals:
##     Min      1Q  Median      3Q     Max 
## -6.3305 -1.8560  0.1309  1.6074  5.6091 
## 
## Coefficients:
##             Estimate Std. Error t value Pr(>|t|)    
## (Intercept) 15.75665    0.60410   26.08   <2e-16 ***
## X2          -1.06114    0.03871  -27.41   <2e-16 ***
## ---
## Signif. codes:  0 '***' 0.001 '**' 0.01 '*' 0.05 '.' 0.1 ' ' 1
## 
## Residual standard error: 2.525 on 48 degrees of freedom
## Multiple R-squared:   0.94,  Adjusted R-squared:  0.9387 
## F-statistic: 751.5 on 1 and 48 DF,  p-value: < 2.2e-16
\end{verbatim}

Erros da regressão de X1

\begin{Shaded}
\begin{Highlighting}[]
\NormalTok{e.X1.X2 <-}\StringTok{ }\KeywordTok{residuals}\NormalTok{(mod.X1.X2)}
\NormalTok{e.X1.X2}
\end{Highlighting}
\end{Shaded}

\begin{verbatim}
##           1           2           3           4           5           6 
##  2.54928419 -0.10160996  1.33771766 -0.15630880  2.68820818  1.47452175 
##           7           8           9          10          11          12 
## -1.43059754 -1.46925673 -4.24080716  1.62913717  1.13638996  0.62807139 
##          13          14          15          16          17          18 
## -3.29582679 -2.63998703 -0.24107028 -2.35524914 -0.06749673  3.65204961 
##          19          20          21          22          23          24 
##  1.83825420 -3.20293645 -2.24240418  2.49856321  1.77681942 -0.70727586 
##          25          26          27          28          29          30 
##  1.27871284  1.06862519 -6.33045544  2.41165748  5.60914387 -1.98493637 
##          31          32          33          34          35          36 
##  1.48884579 -0.94673421 -2.72462001  4.28665523 -2.70586482  3.06723341 
##          37          38          39          40          41          42 
##  0.47206196 -2.03819503  1.54224996 -5.89047866  2.09191129  3.34870860 
##          43          44          45          46          47          48 
## -0.66856885 -1.13382161 -2.29725490  1.04716396  0.69618540  0.32931024 
##          49          50 
## -0.93617323 -0.13955216
\end{verbatim}

\begin{Shaded}
\begin{Highlighting}[]
\KeywordTok{cor}\NormalTok{(X1,e.X1.X2)}
\end{Highlighting}
\end{Shaded}

\begin{verbatim}
## [1] 0.2450241
\end{verbatim}

\begin{Shaded}
\begin{Highlighting}[]
\KeywordTok{cor}\NormalTok{(X2, e.X1.X2)}
\end{Highlighting}
\end{Shaded}

\begin{verbatim}
## [1] 6.756291e-17
\end{verbatim}

Gerando Y1 como uma função de X1

\begin{Shaded}
\begin{Highlighting}[]
\NormalTok{Y1 <-}\StringTok{ }\OperatorTok{-}\DecValTok{9} \OperatorTok{+}\FloatTok{1.3}\OperatorTok{*}\NormalTok{X1 }\OperatorTok{+}\KeywordTok{rnorm}\NormalTok{(}\DecValTok{50}\NormalTok{,}\DataTypeTok{mean=}\DecValTok{0}\NormalTok{,}\DataTypeTok{sd=}\DecValTok{4}\NormalTok{)}
\NormalTok{Y1}
\end{Highlighting}
\end{Shaded}

\begin{verbatim}
##  [1]  -5.0122138  -1.7770927 -17.7147386 -19.3744768   4.6082992
##  [6]  -8.4806440  -4.8835792 -20.7057716 -43.5719344 -10.5447757
## [11] -15.6000416  -4.6552881   2.5662396 -26.2805127   1.6079348
## [16] -10.1215981  15.5814420  -2.8564117  10.3923359  -9.3791057
## [21]   7.0029597 -12.1095065 -11.7798183   2.4741152  19.4663352
## [26] -18.9310700 -25.7904611  21.3712238  -2.7158511 -28.0262280
## [31]  23.3892750   8.1173120 -31.3029595 -20.5438204   1.6857843
## [36]   3.6440067   7.6551513 -12.1736528  -0.2308710  -6.3055604
## [41]   4.9966393  -3.3334495   0.5840852 -16.3826613   3.1847536
## [46] -33.3680393 -11.3365878   1.7723562  -5.7300108 -15.0994689
\end{verbatim}

Modelo explicando Y1 com X1 e X2

\begin{Shaded}
\begin{Highlighting}[]
\NormalTok{mod.Y1.X1X2 <-}\StringTok{ }\KeywordTok{lm}\NormalTok{(Y1 }\OperatorTok{~}\StringTok{ }\NormalTok{X1 }\OperatorTok{+}\StringTok{ }\NormalTok{X2)}
\KeywordTok{summary}\NormalTok{(mod.Y1.X1X2)}
\end{Highlighting}
\end{Shaded}

\begin{verbatim}
## 
## Call:
## lm(formula = Y1 ~ X1 + X2)
## 
## Residuals:
##     Min      1Q  Median      3Q     Max 
## -7.6834 -3.1912  0.5708  3.4546  8.9019 
## 
## Coefficients:
##             Estimate Std. Error t value Pr(>|t|)    
## (Intercept) -16.7684     4.4079  -3.804  0.00041 ***
## X1            1.7513     0.2704   6.477 5.08e-08 ***
## X2            0.4964     0.2959   1.677  0.10012    
## ---
## Signif. codes:  0 '***' 0.001 '**' 0.01 '*' 0.05 '.' 0.1 ' ' 1
## 
## Residual standard error: 4.729 on 47 degrees of freedom
## Multiple R-squared:  0.8936, Adjusted R-squared:  0.8891 
## F-statistic: 197.4 on 2 and 47 DF,  p-value: < 2.2e-16
\end{verbatim}

Modelo explicando Y1 com X1 e o erro do primeiro modelo

\begin{Shaded}
\begin{Highlighting}[]
\NormalTok{mod.Y1.X1e1 <-}\StringTok{ }\KeywordTok{lm}\NormalTok{(Y1 }\OperatorTok{~}\StringTok{ }\NormalTok{X1 }\OperatorTok{+}\StringTok{ }\NormalTok{e.X1.X2)}
\KeywordTok{summary}\NormalTok{(mod.Y1.X1e1)}
\end{Highlighting}
\end{Shaded}

\begin{verbatim}
## 
## Call:
## lm(formula = Y1 ~ X1 + e.X1.X2)
## 
## Residuals:
##     Min      1Q  Median      3Q     Max 
## -7.6834 -3.1912  0.5708  3.4546  8.9019 
## 
## Coefficients:
##             Estimate Std. Error t value Pr(>|t|)    
## (Intercept) -9.39814    0.68858 -13.649   <2e-16 ***
## X1           1.28353    0.06833  18.784   <2e-16 ***
## e.X1.X2      0.46775    0.27887   1.677      0.1    
## ---
## Signif. codes:  0 '***' 0.001 '**' 0.01 '*' 0.05 '.' 0.1 ' ' 1
## 
## Residual standard error: 4.729 on 47 degrees of freedom
## Multiple R-squared:  0.8936, Adjusted R-squared:  0.8891 
## F-statistic: 197.4 on 2 and 47 DF,  p-value: < 2.2e-16
\end{verbatim}

Modelo explicando Y1 com X1 e o erro do primeiro modelo

\begin{Shaded}
\begin{Highlighting}[]
\NormalTok{mod.Y1.X2e1 <-}\StringTok{ }\KeywordTok{lm}\NormalTok{(Y1 }\OperatorTok{~}\StringTok{ }\NormalTok{X2 }\OperatorTok{+}\StringTok{ }\NormalTok{e.X1.X2)}
\KeywordTok{summary}\NormalTok{(mod.Y1.X2e1)}
\end{Highlighting}
\end{Shaded}

\begin{verbatim}
## 
## Call:
## lm(formula = Y1 ~ X2 + e.X1.X2)
## 
## Residuals:
##     Min      1Q  Median      3Q     Max 
## -7.6834 -3.1912  0.5708  3.4546  8.9019 
## 
## Coefficients:
##             Estimate Std. Error t value Pr(>|t|)    
## (Intercept) 10.82591    1.13160   9.567 1.31e-12 ***
## X2          -1.36200    0.07251 -18.784  < 2e-16 ***
## e.X1.X2      1.75128    0.27037   6.477 5.08e-08 ***
## ---
## Signif. codes:  0 '***' 0.001 '**' 0.01 '*' 0.05 '.' 0.1 ' ' 1
## 
## Residual standard error: 4.729 on 47 degrees of freedom
## Multiple R-squared:  0.8936, Adjusted R-squared:  0.8891 
## F-statistic: 197.4 on 2 and 47 DF,  p-value: < 2.2e-16
\end{verbatim}


\end{document}
